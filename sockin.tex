\bold{One may also test our prediction more directly if one can measure the demand fundamental, $A$, of a tokenized platform. Our theory suggests that total transaction fees, which are based on users’ average convenience yield, represent a reliable proxy.}

\section{Notation}

fixed supply: decentralized consensus credibly fixes the token supply and this anchors the token price to user demand
network effects: the more users the platform has, the more useful the tokens are
first-best: Refers to the optimal outcome that can be achieved in an ideal setting. It represents the highest level of welfare or utility attainable according to a given utility function. It serves as a benchmark for evaluating the efficiency of real-world situations or policy interventions. Reality deviates from the ideal setting of the model, preventing first-best outcomes, leading to inefficiencies that economists aim to address.


assumed and derived characteristics
  - discrete-time, infinitely many periods
  - each period, a new generation of users chooses whether to join the platform by purchasing tokens from both existing token holders and from new token issuance by the platform
  - users and speculators holding different beliefs about the capital gain from holding the token
  - monetary nonneutrality (of the token price) together with the network effect in user participation, induces platform fragility. nonneutrality of the token price means, as token price fluctuates, the platform does not adjust the number of tokens required for each user to qualify for the platform's matching services
  - the token price is endogenous (derived)
  - price is set by market clearing in a secondary token market
  - three types of platform participants: users, speculators, and validators
  - every participant is small and does not internalize the effects of her trading on others
  - no issues of trust on the platform
  - user $i$ has a Cobb-Douglas utility function over consumption of his own good and the good of user $j$ a Cobb-Douglas utility function with quasi-linearity in wealth means users are risk-neutral with respect to the token's capital gain
  - $Q_t$ contains information about $A_{t+1}$ , and thus is useful to users for forming their expectations about the token price in the next period, $P_{t+1}$.
  - platforms that rely on network effects (each user's desire to join the platform grows with the number of other users on the platform and the size of their goods endowments).
  - facilitate decentralized bilateral transactions of certain goods or services among a pool of users
  - can trade their goods with each other only if they both belong to the platform
  - a user's benefit from using the token is increasing in the quantity, rather than in value in fiat currency, of tokens that she holds
  - stickiness in the platform in adjusting the number of tokens required for its services in response to token price fluctuation
  - supply of tokens to users is decentralized in that token issuance follows a predetermined schedule, $\Phi\left(y_{t}\right)=\,\Phi\left(y_{t-1}+\iota\right)$, where $\Phi\left(.\right)$ is the normal cumulative distribution function. All key qualitative results are unchanged, if token supply is capped at some finite maximum.
  - token market participants are atomistic and therefore do not internalize how their trading impacts others
  - In deciding whether to join the platform, a user trades off
    - the cost of buying a token with 
    - the benefits from both transacting goods on the platform and 
    - expected token price appreciation
  - users have quasi-linear expected utility 
  - a fixed participation cost $\kappa > 0$ if they choose to join the platform (pecuniary or mental, e.g. the cost of setting up a wallet and installing the software necessary)
  - each user needs to give a fraction $\beta$ of his utility surplus from the transaction as service fee to the platform
  - Each user optimally adopts a cutoff strategy to join the platform by purchasing the token only if its goods endowment is higher than a threshold.
  - threshold and the token price are jointly determined
    - by users' token demand, which is based on their common goods endowment and optimism about token price appreciation and 
    - the net supply of tokens by speculators, which is also determined by their sentiment about token price appreciation.

we examine token prices and platform performance with a realistic information structure that allows us to examine the role of optimism among users and sentiment among speculators

platform serves to reduce search frictions among a pool of users who share a certain need to transact goods with each other

Before finding a transaction partner on the platform, each user needs to decide whether to join
the platform by buying the token

Platform life-cycle implications: more mature platforms might be more vulnerable to market
breakdown, younger platforms might have higher market capitalizations, and token price volatility
is increasing over time.

the persistence of the two return components $\frac{\kappa}{P_{t}}$ and $\frac{\left(1-\beta\right)U_{t}^{*}}{P_{t}}$ can lead to a positive autocorrelation in the expected capital gain. Our model therefore predicts that measures of mining
costs should have more predictive power for the capital gain when there is a nontrivial chance of
strategic attacks, such as when the hashrate or the number of miners is low.

Nonfundamental shocks to token prices, represented by user optimism and speculator sentiment in our model, can also help explain reversal in cryptocurrency returns

In explaining the time series of cryptocurrency price appreciation, this model predicts: 1) a role for both news and investor sentiment, not through risk premia but by predicting the marginal user's convenience yield; 2) a size effect in the capital gain of cryptocurrencies; 3) momentum patterns observed in token price appreciation, through the persistence of user participation costs and convenience yields; 4) the size effect that is observed in the cross-section of cryptocurrency price appreciation; 5) measures of mining/validator costs should have more predictive power for the capital gain when there is a nontrivial chance of strategic attacks. 

Only the expected token price appreciation is determined by the marginal user's equilibrium condition - it equals the total cost of capital and participation minus the convenience yield from transaction surplus.  The convenience yield is created by shareholders acting in their dual capacity as users of the platform, which gives rise to a feedback mechanism from the cryptocurrency return to user participation. The expected excess capital gain does not exhibit conventional risk premia. The capital gain may still exhibit predictability through the underlying state variables that explain the convenience yield. 

A token with deterministic time-varying supply serves as an investable asset for users and speculators, who may speculate about the growth of the platform. Nonfundamental fluctuations in the token price come from users and speculators. User optimism about token price appreciation may be crowded out by speculator sentiment, raising the cost for users to participate. 

Market Breakdown
It is important to recognize that this breakdown is a result of two key features of our model. First, token price fluctuations have a real effect on user participation because the platform does not adjust the number of tokens required for users to participate
The second key feature for market breakdown is that the token market is decentralized and no participant in the market internalizes the effect of her trading on others
  - On the user side, no user accounts for the network effect of her participation choice on other users. 
  - On the speculator side, each speculator takes the token price as given, which implies that when the token market fails to find a market-clearing price, neither is a single speculator present nor can a group of speculators coordinate with each other to offer a price to clear the users' token demand

However, elastic token issuance mitigates this fragility. Increasing the supply of tokens deterministically over time, introduces life-cycle effects for the token's current convenience yield and expected capital gains, which jointly determine the total token return to each user. The inflation of the token base over time lowers expected capital gains. As a result, the region of market breakdown and the relative weight of the convenience yield in the total token return increase over time. Both of these effects, in turn, raise the sensitivity of the user base to the current demand fundamental and log token price volatility over time. Hence, a

  - Utility tokens (Ether, Filecoin, GameCredits): are native currencies accepted on decentralized digital platforms that often provide intrinsic benefit to participants.
  - Coins and altcoins (Bitcoin and Litecoin): are fiat currencies that are maintained on a public blockchain ledger by a decentralized population of record keepers. Coins are typically created through "forks" from existing currencies, and by airdrops
  - Security tokens: are financial assets that trade in secondary markets on exchanges and whose initial sale is recorded on the blockchain of the currency that the issuer accepts as payment. Security tokens are typically sold through ICOs structured as "smart contracts" on existing blockchains such as that of Ethereum

model approach
  - rational expectations
  - structural (explicit utility formulation)
  - user's transaction need is determined by its endowment in a consumption good,
  - motivated by the nonneutrality of money that underlies modern monetary theory

solution approach
while the framework is inherently nonlinear, it yields a tractable cutoff equilibrium. For the marginal user that purchases the token, their optimal endogenous cutoff, $A_t^{*}$, in each period, is provided by the fixed-point condition

\begin{equation}
    \exp{\left(-\frac{\sqrt{r_{\varepsilon}}}{\lambda}\left(A_{t}^{*}-A_{t}\right)-\frac{1}{\lambda}y_{t}+\frac{1}{\lambda}\zeta t\right)}.
\end{equation}

 
Users: overlapping generations of users, In each period, a new generation of users purchase the cryptocurrency as the membership to the platform, and then are randomly matched with each other to transact their goods endowments. Users are effectively also shareholders in the platform through the retradability of the token.

Validators: goods transactions are supported by validators of the decentralized platform who act as service providers and complete all user transactions. Validators have no use for tokens and, potentially for liquidity reasons, sell them immediately to speculators. payment to validators in period t is both the seignorage from the scheduled inflation of the token base and the transaction fees from users.

Speculators: 

\begin{equation}
    \pi_{t}=\left(\Phi\left(y_{t-1}+\iota\right)-\Phi\left(y_{t-1}\right)\right)P_{t}+\beta U_{t},
\end{equation}

where $\beta$ a fraction and $U_t$ is the total transaction surplus on the platform,

\begin{equation}
    U_{t}=e^{A_{t}+\frac{1}{2}((1-\eta_{c})^{2}+\eta_{c}^{2})\tau_{\epsilon}^{-1}}\Phi\left((1-\eta_{c})\,\tau_{\varepsilon}^{-1/2}+\frac{A_{t}-A_{t}^{*}}{\tau_{\varepsilon}^{-1/2}}\right)\Phi\left(\eta_{c}\tau_{\varepsilon}^{-1/2}+\frac{A_{t}-A_{t}^{*}}{\tau_{\varepsilon}^{-1/2}}\right)
\end{equation}

where $\eta_{c} \in (0, 1)$ represents the weight in the Cobb-Douglas utility function on a user's consumption of her trading partner's good, and $1 - \eta_{c}$ is the weight on consumption of her own good. A higher $\eta_{c}$ indicates a stronger complementarity between the consumption of the goods obtained from trading.

With the addition of mining and strategic attacks, the miners' common mining efficiency $\xi_t$ becomes an additional state variable, measured by inversely parameterizing the miner's cost of mining. Strategic attacks occur when either $A_t$ or $\xi_t$ falls below a critical boundary.  This boundary is explored in detail, and it may be possible for both a no-attack equilibrium and an attack equilibrium to be self-fulfilling.

 This arises because both the benefit $\left(\Phi\left(y_{t}+\psi l\right)-\Phi\left(y_{t}\right)\right)P_{t}$ and the cost $x_{t} - {\alpha {e^{2\xi_{t}}}} x_{t}^{2}$
of an attack are positively correlated

Similarly, under the
proof of stake protocol, the most prevalent type of strategic attack is a Sybil attack. Under a Sybil
attack, a rogue validator can acquire 51\% of all staked tokens and create false validator nodes
to manipulate consensus on the blockchain to engage in a distributed denial of service (DDoS)
or "double-spending" attack. Like the 51\% attack under PoW, such a strategic attack is more
difficult when the revenue from validating transactions is higher and there are a lot of tokens
staked to compete for this revenue. Furthermore, a Sybil attack is also harder when the token price
is high because acquiring a 51\% stake size is more expensive. As in our analysis, an attack is more
likely to occur when the platform is weak and user participation is low. By similar logic to our
51\% attack analysis, anticipation of a Sybil attack also reduces user incentives to join the platform,
which increases the region of market breakdown and strategic attacks. Consequently, our analysis
of strategic attacks applies more generally to vulnerabilities of consensus protocols.

Design Implications
Our analysis raises a key issue that the network effect endemic to utility token platforms can lead to fragility when a rigid token supply curve interacts with a demand curve that is subject to a network effect and nonneutrality of the token price.
  - policies that make the supply of tokens respond to speculative shocks, such as a state-contingent token issuance schedule, (governments face similar issues in setting monetary policy, for instance, when deciding whether monetary policy should respond to stock or housing market fluctuations).
  - policies that subsidize users, such as a state-contingent transaction fee rate ($\beta$ in the model), can mitigate the risk of market breakdown.  A state-contingent transaction fee rate may also potentially mitigate market breakdown by adjusting how much transaction surplus users need to give up to compensate the platform's validators. it may be difficult in practice to condition
platform policy on unobservable speculator sentiment in secondary markets
