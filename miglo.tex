@article{miglo22,
       title    = {Theories of Crowdfunding and Token Issues: A Review},
       author   = {Miglo, Anton},
       year     = 2022,
       month    = {May},
       journal  = {Journal of Risk and Financial Management},
       publisher = {Mdpi Ag},
       volume   = 15,
       number   = 5,
       pages    = 218,
       doi      = {10.3390/jrfm15050218},
       issn     = {1911-8074},
       url      = {http://dx.doi.org/10.3390/jrfm15050218},
       abstract = {Entrepreneurial, innovative and smalland medium-sized firms experience difficulties with raising funds using traditional debt and equity. Consequently, they are constantly looking for new strategies of financing. The latest inventions are crowdfunding and token issues. In contrast to traditional ways of raising funds these innovations: (1) use modern technology (online transactions, blockchain, etc.) much more actively; (2) are usually quicker in reaching potential investors/funders; (3) use more active network benefits such as, for example, a large number of interactions between investors/funders and between funders and firms. These changes are so significant that some experts list them among the top business inventions of the 21st century. This article provides a review of the growing number of theoretical papers in the areas of crowdfunding and token issues, compares their findings with empirical evidence and discusses directions for future research. The research shows that a large gap exists between the theoretical literature and empirical literature.},
       annote   = {
        Outline: price model \% state variable \% assumed and derived characteristics \% model approach \% solution approach

        In this article, Miglo provides a comprehensive review of theories related to crowdfunding and token issues in the context of the evolving financial landscape. The author examines various theoretical frameworks and concepts to understand the dynamics and implications of crowdfunding campaigns and token offerings.

        The article explores different theoretical perspectives, including agency theory, signaling theory, and information asymmetry, to analyze the motivations and behaviors of investors and entrepreneurs in the crowdfunding space. Miglo also discusses the emergence of tokenization and the use of blockchain technology in fundraising activities, highlighting its potential impact on traditional financing channels.

        Through a systematic review of existing literature, the author presents a synthesis of key theories and findings in the field of crowdfunding and token issues. The analysis contributes to a deeper understanding of the economic, behavioral, and financial aspects of these fundraising mechanisms.

        Miglo's article is valuable for researchers, practitioners, and policymakers interested in gaining insights into the theoretical foundations of crowdfunding and tokenization. It provides a solid overview of the conceptual frameworks that underpin these areas, facilitating further research and discussion in the field of alternative finance.

        Overall, Miglo's review article offers a comprehensive and well-organized analysis of theories related to crowdfunding and token issues. It serves as a valuable resource for readers seeking a deeper understanding of the theoretical underpinnings and implications of these innovative fundraising models.
    },
}

%
@article{cong321,
    %     author = {Cong, Lin William and He, Zhiguo and Li, Jiasun},
    %     title  = {Decentralized Mining in Centralized Pools},
    %     journal = {The Review of Financial Studies},
    %     volume = 34,
    %     number = 3,
    %     pages  = {1191-1235},
    %     year   = 2021,
    %     month  = {04},
    %     abstract = {The rise of centralized mining pools for risk sharing does not necessarily undermine the decentralization required for blockchains: because of miners' cross-pool diversification and pool managers' endogenous fee setting, larger pools better internalize their externality on global hash rates, charge higher fees, attract disproportionately fewer miners, and grow more slowly. Instead, mining pools as a financial innovation escalate miners' arms race and significantly increase the energy consumption of proof-of-work-based blockchains. Empirical evidence from Bitcoin mining supports our model's predictions. The economic insights inform other consensus protocols and the industrial organization of mainstream sectors with similar characteristics but ambiguous prior findings.},
    %     issn   = {0893-9454},
    %     doi    = {10.1093/rfs/hhaa040},
    %     url    = {https://doi.org/10.1093/rfs/hhaa040},
    %     annote = {
        Outline: price model \% state variable \% assumed and derived characteristics \% model approach \% solution approach

        \% Intuitively, larger pools have more \% market power because the risk-sharing benefit it provides is larger. \% Therefore, pool owners charge higher fees, leading to a smaller percent- \% age growth in pool size. Empirical evidence supports the theoretical pre- \% dictions. Every quarter, the authors sort pools into deciles based on the \% start-of-quarter pool size and calculate the average pool share, average \% fee, and average log growth rate for each decile. They show that pools \% with larger start-of-quarter size charge higher fees and grow slower in \% percentage terms. They investigate these relationships in three 2-year \% spans (i.e., 2012–2013, 2014–2015, and 2016–2017, as shown in Figure 3) \% and find that almost all of them are statistically significant with the signs \% predicted by their theory. \% The insights from this chapter can be extended to other protocols such \% as PoS, because miners in PoS systems also form coalitions (e.g., Brunjes \% et al., 2018).
    },
}
%
@article{sockin23a,
    %   title    = {Decentralization through Tokenization},
    %   author   = {Sockin, Michael and Xiong, Wei},
    %   year     = 2023,
    %   journal  = {The Journal of Finance},
    %   volume   = 78,
    %   number   = 1,
    %   pages    = {247--299},
    %   doi      = {10.1111/jofi.13192},
    %   url      = {https://dx.doi.org/10.1111/jofi.13192},
    %   abstract = {We examine decentralization of digital platforms through tokenization as an innovation to resolve the conflict between platforms and users. By delegating control to users, tokenization through utility tokens acts as a commitment device that prevents a platform from exploiting users. This commitment comes at the cost of not having an owner with an equity stake who, in conventional platforms, would subsidize participation to maximize the platform's network effect. This trade-off makes utility tokens a more appealing funding scheme than equity for platforms with weak fundamentals. The conflict reappears when nonusers, such as token investors and validators, participate on the platform.},
    %   annote   = {
        Outline: price model \% state variable (information structure) \%   - the aggregate transaction surplus of users \%      - in terms of information structure, this is the only fundamental \% assumed and derived characteristics \%   - the token price is endogenous (derived) \%   - price is set by the developer who controls the supply of tokens \% model approach \% solution approach

        \% focus is on platform governance

        \% empirical implications: explain cross-sectional patterns in ICOs

        \% highlight that tokenization helps to decentralize the control of the platform and makes it possible for the platform to commit to not exploiting its users, at the expense of not having an owner with an equity stake to subsidize user participation and maximize the platform's network effect

        \% formally show that, consumption allocations between two paired users can be microfounded through a trading mechanism between them

        \% In this article, Sockin and Xiong delve into the concept of decentralization through tokenization, exploring the role of tokens in enabling decentralized governance and economic systems. The authors examine how tokens can be used to facilitate peer-to-peer transactions, incentivize network participation, and allocate decision-making power within decentralized platforms.

        \% The article presents a theoretical framework that captures the economic and governance implications of tokenization. Sockin and Xiong analyze the conditions under which tokens can effectively enable decentralization, considering factors such as network effects, information asymmetry, and incentive alignment. The research also highlights the potential challenges and risks associated with tokenized systems, including issues of liquidity, market manipulation, and regulatory considerations.

        \% The findings presented in this article have important implications for the understanding of decentralized finance (DeFi) and the broader adoption of tokenized systems. Sockin and Xiong's analysis contributes to the ongoing discussion on the potential benefits and drawbacks of tokenization as a mechanism for decentralization and innovation in financial markets.

        \% The article is relevant for researchers, practitioners, and policymakers interested in the intersection of blockchain technology, decentralized governance, and finance. It provides a comprehensive analysis of the economic and governance aspects of tokenization and offers valuable insights into the potential role of tokens in reshaping financial systems.

        \% Overall, Sockin and Xiong's article on "Decentralization through Tokenization" provides a rigorous examination of the economic and governance implications of tokenized systems. It is a significant contribution to the literature on decentralized finance and token economics, offering theoretical insights and practical considerations for the future development of decentralized platforms.
    },
}
% volume       = {0},
  % number       = {0},
  % pages        = {null},
